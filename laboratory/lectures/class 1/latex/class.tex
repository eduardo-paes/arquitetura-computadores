\documentclass[10pt]{article}
\usepackage[applemac]{inputenc}
\usepackage{caption}
\usepackage{graphicx}
\usepackage{latexsym}
\usepackage{listings}
\usepackage{amsmath}
\usepackage{multirow}
\usepackage{multicol}
\usepackage{amsfonts}
\usepackage{amssymb}
\usepackage{epsfig}
\usepackage[left=2cm,top=2cm,right=2cm,bottom=2cm]{geometry}
\usepackage[lined, ruled, vlined]{algorithm2e}
\usepackage[normalem]{ulem}
\usepackage[abs]{overpic}
\usepackage{hyperref}

\newcounter{QuestionCounter}
\setcounter{QuestionCounter}{1}

\begin{document}

\begin{center}
\rule{18.1cm}{0.2mm} \\
\end{center}
\begin{tabular}{lc}
\thispagestyle{empty}
\multirow{4}{25mm}{\includegraphics[width=1.5cm]{logo.eps}}  
\hspace{0.4cm} & {\bf \Large Computer Architecture } \\
\hspace{0.4cm} & {\bf Professor Lu�s Tarrataca}\\
\hspace{0.4cm} & {\bf Laboratory 1}\\
\hspace{0.4cm} & {Centro Federal de Educa\c c�o Tecnol�gica Celso Suckow da Fonseca}\\
\end{tabular}\\ \\
\centerline{\rule{18.1cm}{0.2mm} \\}
\vspace{0.1cm}


%\begin{multicols*}{2}

\noindent{\Large \bf Objectives:}

\vspace{0.25cm} 
Introduction to 

\begin{itemize}
		
	\item P3 processor;
	
	\item P3 programming language:
	
	\begin{itemize}
		
		\item data transfer instructions;
		
		\item address modes;

	\end{itemize}
	
	\item P3 simulator:
	
	\begin{itemize}
		
		\item Execution of the environment;
		
		\item Testing the program;

	\end{itemize}

\end{itemize}

\vspace{0.25cm} 
\noindent{\Large \bf P3 Processor:}

\vspace{0.25cm} 
The P3 processor was developed at the \href{http://www.ulisboa.pt/}{University of Lisbon} - \href{https://tecnico.ulisboa.pt/en/}{Instituto Superior T�cnico} for the students of the \href{https://fenix.tecnico.ulisboa.pt/disciplinas/IAC45179/2016-2017/1-semestre}{``Computer Architecture''} course. The P3 is a 16-bit RISC processor containing 8 generic registers, allows for the normal set of arithmetic, logic and control instructions. The processor also has several mechanism for processing inputs (\textit{e.g.:} reading from the keyboard) and outputs (\textit{e.g.} printing to a screen). It also features an interruption processing mechanism as well as a microprogramming component. All of the informations regarding the processor can be found on the manual that is provided for the laboratories.

\vspace{0.5cm} 
\noindent{\Large \bf P3 Simulator and Assembly:}

\vspace{0.25cm} 

For this section we will be using the file ``lab1.as'':

\begin{enumerate}

	\item To compile the assembly code the command ``./p3as-linux lab1.as'' needs to executed on the terminal;
	
	\item To run the simulator the command ``java -jar p3sim.jar'' needs to be executed on the terminal;
	
	\begin{enumerate}
	
		\item To load the compiled binary file choose ``Ficheiro $\rightarrow$ Carrega Programa'';
		
		\item Before you execute the program load the I/O window by choosing ``Ver $\rightarrow$ Janela de Texto '';
		
		\item Execute the program by pressing the button ``Corre''
		
	\end{enumerate}
	
	\item Examine the code;

\end{enumerate}







\end{document}